\documentclass{article}
\usepackage{multicol}
\usepackage{multirow}
\usepackage{graphicx}
\usepackage{rotating}
\usepackage{hyperref}
\def\degrees{{\small$^{\mathrm{o}}$}}

\begin{document}
\listoftables
\listoffigures
\section{Scalings, Rotations using \texttt{graphics}}

\begin{table}[htb]
\centering
  \begin{tabular}{|c|c|c|c|c|} \hline
   \multicolumn{1}{|c|}{\multirow{5}{5pt}{\begin{rotatebox}{90}{{\footnotesize \textbf{Original pairs}}}\end{rotatebox}}} &
     \multicolumn{1}{|c|}{ } & \multicolumn{3}{|c|}{\textbf{Monothematic pairs}}\\ \cline{2-5}
     & & Y & N & Total \\ \cline{2-5}
     & Y (32) & 116 & \ -- &  116  \\ \cline{2-5}
     & N (29) & \ \ 96 & 29 & 125  \\ \cline{2-5}
     & Total (61) & 212 & 29 & 241  \\ \hline
  \end{tabular}
  \caption{Normal: Distribution of monothematic pairs with respect to original Y/N pairs}
  \label{tb:tab1}
\end{table}

\begin{table}[htb]
\scalebox{1.5}[0.75]{
\centering
  \begin{tabular}{|c|c|c|c|c|} \hline
   \multicolumn{1}{|c|}{\multirow{5}{5pt}{\begin{rotatebox}{90}{{\footnotesize \textbf{Original pairs}}}\end{rotatebox}}} &
     \multicolumn{1}{|c|}{ } & \multicolumn{3}{|c|}{\textbf{Monothematic pairs}}\\ \cline{2-5}
     & & Y & N & Total \\ \cline{2-5}
     & Y (32) & 116 & \ -- &  116  \\ \cline{2-5}
     & N (29) & \ \ 96 & 29 & 125  \\ \cline{2-5}
     & Total (61) & 212 & 29 & 241  \\ \hline
  \end{tabular}
  }
   \caption{Squinched: Distribution of monothematic pairs with respect to original Y/N pairs}
  \label{tb:tab2}
\end{table}

\begin{table}[htb]
\reflectbox{
\centering
  \begin{tabular}{|c|c|c|c|c|} \hline
   \multicolumn{1}{|c|}{\multirow{5}{5pt}{\begin{rotatebox}{90}{{\footnotesize \textbf{Original pairs}}}\end{rotatebox}}} &
     \multicolumn{1}{|c|}{ } & \multicolumn{3}{|c|}{\textbf{Monothematic pairs}}\\ \cline{2-5}
     & & Y & N & Total \\ \cline{2-5}
     & Y (32) & 116 & \ -- &  116  \\ \cline{2-5}
     & N (29) & \ \ 96 & 29 & 125  \\ \cline{2-5}
     & Total (61) & 212 & 29 & 241  \\ \hline
  \end{tabular}
  }
   \caption{Flipped: Distribution of monothematic pairs with respect to original Y/N pairs}
  \label{tb:tab3}
\end{table}

\begin{table}[htb]
\rotatebox{45}{
\centering
  \begin{tabular}{|c|c|c|c|c|} \hline
   \multicolumn{1}{|c|}{\multirow{5}{5pt}{\begin{rotatebox}{90}{{\footnotesize \textbf{Original pairs}}}\end{rotatebox}}} &
     \multicolumn{1}{|c|}{ } & \multicolumn{3}{|c|}{\textbf{Monothematic pairs}}\\ \cline{2-5}
     & & Y & N & Total \\ \cline{2-5}
     & Y (32) & 116 & \ -- &  116  \\ \cline{2-5}
     & N (29) & \ \ 96 & 29 & 125  \\ \cline{2-5}
     & Total (61) & 212 & 29 & 241  \\ \hline
  \end{tabular}
  }
   \caption{Flipped: Distribution of monothematic pairs with respect to original Y/N pairs}
  \label{tb:tab3}
\end{table}

\section{Scalings, Rotations using \texttt{rotating}}

Start here
\begin{rotate}{-56}
Save whales
\end{rotate}
End here

Start here \begin{turn}{56}%
Save the whale
\end{turn} end here

Start here
\begin{sideways}%
Save the whale
\end{sideways}
End here

\newsavebox{\foo}
\def\testrot#1{%
\savebox{\foo}{\parbox{1in}{Save 
the whales Save the whale Save the \href{http://dlmf.nist.gov/LaTeXML/}{Kitty!} Save the whale}}%
\framebox{---\begin{turn}{#1}\framebox{\usebox{\foo}}\end{turn}---}}%

\begin{figure*}
\begin{tabular}{|c|c|c|}
\hline
\testrot{0} &\testrot{-40}&\testrot{-80}\\
0\degrees & -40\degrees & -80\degrees \\
\hline
\testrot{-120}&\testrot{-160}&\testrot{-200}\\
120\degrees & -160\degrees & -200\degrees \\
\hline
\testrot{-240}&\testrot{-280}&\testrot{-320}\\
-240\degrees & -280\degrees & -320\degrees \\
\hline
\end{tabular}
\caption{Rotation of paragraphs between 0 and -320 degrees \label{angles1}}
\end{figure*}

\begin{quote}
\rule{0pt}{1.5in}\begin{tabular}{rrr}
\begin{rotate}{45}Column 1\end{rotate}&
\begin{rotate}{45}Column 2\end{rotate}&
\begin{rotate}{45}Column 3\end{rotate}\\
\hline
1& 2& 3\\
4& 5& 6\\
7& 8& 9\\
\hline
\end{tabular}
\end{quote}

\begin{quote}
\begin{tabular}{rrr}
\begin{turn}{45}Column 1\end{turn}&
\begin{turn}{45}Column 2\end{turn}&
\begin{turn}{45}Column 3\end{turn}\\
\hline
1& 2& 3\\
4& 5& 6\\
7& 8& 9\\
\hline
\end{tabular}
\end{quote}

\begin{sidewaystable}
\centering
\begin{tabular}{|llllllllp{1in}lp{1in}|}
\hline
Context   &Length   &Breadth/   &Depth   &Profile   &Pottery   &Flint   &Animal   &Stone   &Other    &C14 Dates \\
  &         &Diameter   &        &          &          &        & 
Bones&&&\\
\hline
&&&&&&&&&&\\
\multicolumn{10}{|l}{\bf Grooved Ware}&\\
784       &---        &0.9m       &0.18m   &Sloping U &P1       &$\times$46  &  $\times$8      &&       $\times$2 bone&  2150$\pm$ 100 BC\\
785       &---        &1.00m      &0.12    &Sloping U &P2--4    &$\times$23  &  $\times$21     & Hammerstone &---&---\\
962       &---        &1.37m      &0.20m   &Sloping U &P5--6    &$\times$48  &  $\times$57*    & ---&     ---&1990 $\pm$ 80 BC (Layer 4) 1870 $\pm$90 BC (Layer 1)\\
983       &0.83m      &0.73m      &0.25m   &Stepped U &---      &$\times$18  &  $\times$8      & ---& Fired clay&---\\
&&&&&&&&&&\\
\multicolumn{10}{|l}{\bf Beaker}&\\
552       &---        &0.68m      &0.12m   &Saucer    &P7--14   &---           & ---       & ---       &---        &---\\
790       &---        &0.60m      &0.25m   &U         &P15      &$\times$12    & ---       & Quartzite-lump&---    &---\\
794       &2.89m      &0.75m      &0.25m   &Irreg.    &P16      &$\times$3     & ---       & ---       &---        &---\\
\hline
\end{tabular}
 
\caption[Grooved Ware and Beaker Features, their Finds and
Radiocarbon Dates]{Grooved Ware and Beaker Features, their
Finds and Radiocarbon Dates; For a breakdown of the Pottery
Assemblages see Tables I and III; for
the Flints see Tables II and IV; for the
Animal Bones see Table V.}\label{rotfloat2}
\end{sidewaystable}
\end{document}
