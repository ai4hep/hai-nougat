% Plain TeX for a 1 page document

%% The lines between the two rows of %'s are more or less compulsory.
%%%%%%%%%%%%%%%%%%%%%%%%%%%%%%%%%%%%%%%%%%%%%%%%%%%%%%%%%%%%%%%%%%%%%

\magnification=\magstep1
\baselineskip=12pt
\hsize=6.3truein
\vsize=8.7truein
\font\footsc=cmcsc10 at 8truept
\font\footbf=cmbx10 at 8truept
\font\footrm=cmr10 at 10truept
\footline={\footsc the electronic journal of combinatorics
   {\footbf 16} (2009), \#R00\hfil\footrm\folio}

%%%%%%%%%%%%%%%%%%%%%%%%%%%%%%%%%%%%%%%%%%%%%%%%%%%%%%%%%%%%%%%%%%%%%%%%
%% The further structure of the front page need not be exactly as below,
%% but the header must contain the names and addresses of the authors
%% as well as the submission and acceptance dates.

\font\bigrm=cmr12 at 14pt
\centerline{\bigrm An elementary proof of the reconstruction conjecture}

\bigskip\bigskip

\centerline{D. Remifa\footnote*{Thanks to
  the editors of this wonderful journal!}}
\smallskip
\centerline{Department of Inconsequential Studies}
\centerline{Solatido College, North Kentucky, USA}
\centerline{\tt remifa@dis.solatido.edu}

\bigskip

\centerline{\footrm 
Submitted: Jan 1, 2009; Accepted: Jan 2, 2009; Published: Jan 3, 2009}
\centerline{\footrm Mathematics Subject Classifications: 05C88, 05C89}

\bigskip\bigskip

\centerline{\bf Abstract}
\smallskip
{\narrower\noindent
The reconstruction conjecture states that the multiset of unlabeled
vertex-deleted subgraphs of a graph determines the graph, provided it
has at least 3 vertices.  A version of the problem was first stated
by Stanis\l aw Ulam.  In this paper, we show that the conjecture can
be proved by elementary methods.  It is only necessary to integrate
the Lenkle potential of the Broglington manifold over the quantum
supervacillatory measure in order to reduce the set of possible
counterexamples to a small number (less than a trillion).  A simple
computer program that implements Pipletti's classification theorem
for torsion-free Aramaic groups with simplectic socles can then
finish the remaining cases.\par}

\bigskip

\beginsection 1. Introduction.

This is the start of the introduction.

\beginsection 2. Equations

\def\eqlabel{(1)}
\def\draftname{Draft}
$$ a=b+c \eqno{\eqlabel \rlap{\draftname}}$$

$$ a=b+c \eqno (1)$$

$$ a=b+c \eqno (1^2)$$

\beginsection 3. Theorems

\proclaim {1.2.3} A Theorem description.
The body, perhaps proof or whatever.

Now comes new material following the theorem, I would guess.


\bye